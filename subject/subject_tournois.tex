\subsection{Tournois}

Au cours de la finale, des tournois auront lieu à des heures
fixes. Ils verront s'affronter les champions issus de la dernière
soumission de chaque candidat.

Le site de soumission des champions se trouve à l'adresse
\href{http://concours/} ; vous pourrez y téléverser votre code. Ce
dernier devra être empaqueté dans une archive \texttt{.tar} contenant
les fichiers nécessaires à la compilation, que vous pourrez produire
au moyen de la commande \texttt{make tar}.

Des tournois intermédiaires, qui n'auront aucune incidence sur le
classement final, se dérouleront aux horaires suivants :
\begin{itemize}
\item \textbf{Samedi 15~h~42} : Ce premier tournoi servira à tester
  l'interface de soumission. On n'attend pas de vous d'avoir déjà
  implémenté une stratégie sophistiquée !
\item \textbf{Samedi 17~h~42} : Deuxième chance pour tester les
  soumissions. Les tournois intermédiaires suivants sont espacés de 6
  heures.
\item \textbf{Samedi 23~h~42}
\item \textbf{Dimanche 5~h~42}
\item \textbf{Dimanche 11~h~42}
\item \textbf{Dimanche 17~h~42} : Dernier tournoi intermédiaire, qui
  aura lieu 7 heures avant le tournoi final.
\end{itemize}

Suite à chacun de ces tournois intermédiaires, le classement sera
affiché sur le site, ainsi qu'un graphe retraçant l'évolution des
rangs de chaque champion sur les tournois précédents.

Le tournoi final, qui décidera du classement, aura lieu à
\textbf{00h42}. Les codes du rendu final seront relus par le jury,
aussi, nous vous demandons d'y inclure en commentaire une description
de la structure du programme ainsi que de la stratégie mise en œuvre
par votre champion.

