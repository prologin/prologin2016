% !TEX encoding = UTF-8 Unicode
% $Header: /cvsroot/latex-beamer/latex-beamer/solutions/conference-talks/conference-ornate-20min.en.tex,v 1.6 2004/10/07 20:53:08 tantau Exp $

\documentclass{beamer}

\mode<presentation>
{
  \usetheme{Warsaw}
  % or ...

  \setbeamercovered{transparent}
  % or whatever (possibly just delete it)
  
  \setbeamertemplate{navigation symbols}{}
  
  \newcommand*\oldmacro{}%
  \let\oldmacro\insertshorttitle%
  \renewcommand*\insertshorttitle{%
    \oldmacro\hfill%
    \insertframenumber\,/\,\inserttotalframenumber}
}

\usepackage[utf8]{inputenc}
% or whatever

\usepackage{times}
\usepackage{multirow}
\usepackage[T1]{fontenc}
\usepackage[french]{babel}
\usepackage{graphicx}

\usepackage{eso-pic}
\usepackage{color}
\usepackage{tikz}
\usepackage{wasysym}

% Or whatever. Note that the encoding and the font should match. If T1
% does not look nice, try deleting the line with the fontenc.

\title[Petit guide des bonnes pratiques pour la construction et la maintenance d'$\alpha$-extracteurs cosmiques à vacuité]
{}

\titlegraphic{\raisebox{2em}{}}

\author[Prologin]
{\includegraphics{../prologin2016}}

\date
{}

\begin{document}

\definecolor{vert}{rgb}{0.07 0.54 0.07}


\begin{frame}
	\frametitle{Petit guide des bonnes pratiques pour la construction et la maintenance d'$\alpha$-extracteurs cosmiques à vacuité}
	\vspace{0.5cm} \centering \includegraphics[width=0.9\linewidth]{../prologin2016}
\end{frame}

\begin{frame}
	\frametitle{Présentation}
	\begin{itemize}
        \item Jeu de stratégie à 2 joueurs : \textcolor{green}{Vert} et
            \textcolor{red}{Rouge}
        \item Tour par tour
        \item Carte carrée de taille fixe
	\end{itemize}
\end{frame}

\begin{frame}
	\frametitle{Plateau}
\end{frame}

\begin{frame}
	\frametitle{Pulsar}
        \begin{itemize}
        \item Période de pulsation $T$
        \item Puissance d'émission $P$
        \item Nombre de pulsations restantes $R$
	\end{itemize}
\end{frame}

\begin{frame}
	\frametitle{Plasma}
        \begin{itemize}
        \item Ressource émise par les pulsars
        \item Se déplace dans la direction qui qui le raproche le plus d'une base
        \item Disparaît s'il n'est relié à aucun tuyau à la fin d'un tour
	\end{itemize}
\end{frame}

\begin{frame}
	\frametitle{Base}
        \begin{itemize}
        \item 5 points d'aspirations maximum par case
        \item Possède une puissance d'aspiration qui réduit la distance de cette case au reste du plateau
        \item Case de récolte du plasma
	\end{itemize}
\end{frame}

\begin{frame}
	\frametitle{Points d'actions}
	\begin{itemize}
	\item[+] commencer un tour
	\item[\alert{--}] construire un tuyau
	\item[\alert{--}] améliorer un tuyau
	\item[\alert{--}] déplacer la puissance d'aspiration
	\item[\alert{--}] détruire un tuyau/super-tuyau
	\item[\alert{--}] déblayer un tuyau détruit
	\end{itemize}
\end{frame}

\begin{frame}
    \frametitle{Conférences}
\end{frame}
\end{document}
