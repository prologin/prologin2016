\newpage ~

\vspace{5cm}

\noindent\emph{O blas bougriot glabouilleux\\
Tes micturations me touchent\\
Comme des flatouillis slictueux\\
Sur une blotte mouche\\
Grubeux, je t'implore\\
Car mes fontins s'empalindroment\ldots\\
Et surrénalement me sporent\\
De croiçantes épiquarômes.\\
Ou sinon\ldots nous t'échierons dans les gobinapes\\
Du fond de notre patafion\\
Tu verras si j'en suis pas cap!
}

\vspace{1cm}

\hspace{5cm} --- Poème Vogon

\newpage

\section{Introduction}

\emph{It is a mistake to think you can solve any major problems just with
potatoes.}


\vspace{1cm}

\textbf{Bienvenue à Vogogle.}

Si vous lisez ces lignes, c'est que vous faites partie des rares individus de à
avoir passé nos entretiens (avec brio !), et faites enfin partie de notre
grande famille\footnote{Vous êtes maintenant un Vogogler. Parlez-en à vos
amis !}. Félicitations !

Prenez quelques minutes pour méditer sur votre exploit : si vous êtes parvenus
jusqu'ici, c'est que :

\begin{itemize}
    \item vous faites partie des 3 103 274 212 112 ayant postulé en remplissant
        les formulaires 1149, 1459 et 802.11 sur \texttt{RS-123 BETELGEU~1} ;
    \item vous avez successivement passé l'épreuve du feu, l'épreuve de l'eau,
        l'épreuve du vent, l'épreuve de la chaussure qui fait un peu mal au
        gros orteil, la dictée à dos de cheval et le remplissage de formulaire
        peu clair, imprécis et légèrement embarrassant ;
    \item Vous avez découvert la série d'indices qui vous menait à la planète
        SOL3, en prenant la première à gauche après Proxima ;
    \item Vous avez trouvé le faux concours d'informatique qui nous servait de
        couverture pour dissimuler nos activités sur cette planète primitive ;
    \item Vous avez prouvé que vous faisiez bien partie de nos meilleurs
        candidats en réussissant nos épreuves d'informatique\footnote{Bien
        entendu, ces épreuves étaient bien trop dures pour être résolues par de
        simples humains, il n'y a donc aucun risque que les autochtones soient
        parvenus jusqu'ici.}.
\end{itemize}

Nous vous remercions également d'avoir signé notre fausse autorisation de
diffusion d'image, dont vous aviez bien sûr relevé la mention en
petites\footnote{7 nanomètres de largeur, au dessus du minimum prévu par la
loi.} lignes retranscrite ci-dessous :

\emph{Je m'engage également à compter du au sein de la société Vogogle, en
exerçant les fonctions d'ingénieur-architecte constructeur de niveau 4, à me
conformer aux dispositions conventionnelles en vigueur dans la société Vogogle
pour une durée indéterminée comprenant une période d'engagement
non-interruptible de 2455 années terrestres, en vertu du code du travail en
vigueur dans le système étatique du sous-groupe galactique Gamma.}

Nous imaginons très bien la joie que vous devez ressentir à l'idée de
collaborer avec notre entreprise. Ce petit guide qui vous a été remis a
plusieurs objectifs. Tout d'abord, vous présenter Vogogle, ses différents
aspects, ses valeurs, ses engagements, et tous les avantages qu'elle vous
apportera dans votre parcours professionnel. Ensuite, vous détailler en quoi
consistera votre rôle et vos objectifs pour bien entamer votre carrière parmi
nous.

Bon courage !

\newpage

\section{Présentation de Vogogle}

\subsection{Notre activité}

Fondée il y a 10 732 201 années terrestres par deux frères Vogons, l'entreprise
de construction Vogogle a commencé son activité par de simple terraformations
de planètes recyclables. La qualité des prestations de Vogogle lui a peu à peu
permi de se faire connaître dans les hautes sphères, ce qui lui a permis
d'étendre son activité dans l'assemblage de systèmes planétaires, la mise en
orbite de satellite naturels, les spectacles d'éclipse, la mise en place de
soleils de plaisance ainsi que la destruction de planètes qui gâchent un peu
la vue.

Ainsi, après seulement quelques petits millions d'années d'existence, Vogogle
est devenue l'entreprise de référence en construction spatiale, ce qui lui a
permis de s'attaquer à de nouveaux secteurs d'activités encore inexplorés,
comme l'arrachage chirurgical de branches de galaxies ou le dépannage et
tractation des vaisseaux coincés au delà de l'horizon des événements des trous
noirs de petite et moyenne taille.

\subsection{Notre modèle économique}

Pour faire simple, toutes les études de marché sont claires sur un point :
l'internalisation des externalités rencontrent certaines limites comme la
rationalité limité des agents ou l'incohérence inter temporelle, c'est pourquoi
la fiscalité patrimoniale des institutions politico-internationales légales
s'impose d'elle même par la construction emphasique des inégalités sur le
terrain, à la fois covariantes et contravariantes. Nous tirons ainsi notre
stratégie principalement de la taxation quadratique des agents commerciaux
indépendants, notamment dans le cadre de biens substituables par effet de
concurrence et de découragement. On a ainsi affaire à un problème de TGO
(Théorie Générale des Organisations), pour avoir la vogolexicomatisation des
lois du marché propre par sa nucléarité, vous l'aurez compris.

\subsection{Avantages des associés}

En tant qu'associé de Vogogle, les avantages dont vous disposez sont nombreux :

\begin{itemize}
    \item plusieurs dizaines de minutes de repos par jour\footnote{Pendant
        lesquelles vous serez d'astreinte, bien sûr.} ;
    \item 5 semaines de congés payés par siècle ;
    \item des primes importantes en cas d'accident du
        travail\footnote{Rassurez-vous, ceux-ci sont assez réguliers pour
        que ces compensations soient importantes !} ;
    \item et bien d'autres\footnote{Vous les découvrirez bien assez tôt.} !
\end{itemize}

\subsection{Nos valeurs}

À Vogogle, notre priorité absolue est la productivité. Nous croyons au
sacrifice de soi pour permettre la croissance économique. Nous ne pouvons donc
pas tolérer que nos employés perdent de précieuses minutes d'inattention.
Rassurez-vous : nous avons tout prévu ! Nous installerons dès que possible un
un appareil qui vous permettra de rester concentré et attentif en toutes
circonstances, en vous envoyant régulièrement des décharges éléctriques afin
que chacun de vos neurones reste dédié à la tâche qui vous est confiée. Grâce à
ce système, vous êtes assuré de faire évoluer rapidement votre carrière !

\subsection{Votre mission}

Toute notre activité requiert une énergie importante, et c'est pour cela que
nous vous avons recruté. Voyez-vous, 
